\section{Introduction: bridging data driven and mechanistic
modelling}\label{introduction-bridging-data-driven-and-mechanistic-modelling}}

The 21\textsuperscript{st} century has been characterised as the century
of complexity.\footnote{This quote is attributed to Stephen Hawking, in
  an interview with the San Jose Mercury News in January 2000.} Shifting
social, economic, environmental, and technological forces have created
increasingly interconnected communities, affected by `wicked' problems
in domains such as health, climate, and economics \cite{Rittel-dilemmas73}. This complexity is reflected in today's scientific
agenda: whether in natural, physical, medical, environmental, or social
sciences, researchers are often interested in the dynamics of complex
systems and the phenomena that emerge from them.

Science has always proceeded through the collection of data. Through
their experiments and observations, researchers collect data about the
world, use this data to develop models or theories of how the world
works, make predictions from those models, then test those predictions,
leading to further refinements to the model and the underpinning theory.
Digitisation of daily activities---in the lab, and elsewhere -- means
that researchers today have access to more data from a greater range of
sources than ever before. In parallel, more sophisticated tools to
collect data have opened new scales of scientific inquiry, from detailed
patterns of gene expression to light signals from other galaxies. Data
proliferation is both a signal of the complexity of today's environment,
and an opportunity to make sense of such complexity.

Advances in artificial intelligence (AI) have produced new analytical
tools to make sense of these data sources. The term `AI' today describes
a collection of methods and approaches to create computer systems that
can perform tasks that would typically be associated with `intelligent'
behaviour in living systems.\footnote{While not the only branch of the
  field, machine learning is the approach to AI that has delivered many
  of the recent advances in AI. Machine learning is an approach to AI in
  which models process data, learning from that data to identify
  patterns or make predictions. In this document, the terms machine
  learning and AI are used interchangeably.} In this document, the term
AI is used broadly, to refer to algorithmic decision-making systems that
combine data, mathematical models, and compute power to make predictions
about the world.

AI is already unlocking progress across research disciplines:

\begin{itemize}
\item
  In Earth sciences, it is helping researchers investigate how different
  parts of the Earth's biosphere interact, and are affected by climate
  change.\footnote{These examples are inspired by talks given at the
    Dagstuhl seminar; these are provided later in the document. This
    example is inspired by Markus Reichstein's talk.}
\item
  In climate science, it supports modelling efforts to reconstruct
  historical climate patterns, enabling more accurate predictions of
  future climate variability.\footnote{This example is inspired by Ieva
    Kazlauskaitė's talk.}
\item
  In agricultural science, it is helping farmers access faster diagnoses
  of animal diseases, enabling more effective responses.\footnote{This
    example is inspired by Dina Machuve's talk.}
\item
  In astrophysics, it is advancing understandings of the nature of dark
  matter and its role in the Universe.\footnote{This example is inspired
    by Siddharth Mishra-Sharma's talk.}
\item
  In developmental biology, it is generating insights into the genetic
  processes that shape how cells develop and differentiate into
  specialist roles.\footnote{This example is inspired by Maren Büttner's
    talk.}
\item
  In environmental science, it allows researchers to analyse the
  features of natural environments more accurately, aiding land and
  resource managers.\footnote{This example is inspired by Christian
    Igel's talk.}
\item
  In neuroscience, it can help model how different neural circuits fire
  to deliver different behaviours in animals.\footnote{This example is
    inspired by Jakob Macke's talk.}
\end{itemize}

The diversity of these successes illustrates the transformative
potential of AI for research across the natural, physical, social,
medical, and computer sciences, arts, humanities, and engineering. By
enabling researchers to extract insights from a greater volume of data,
drawn from a wider variety of sources, and operating across multiple
dimensions and scales, AI could unlock new understandings of the world.
In so doing, AI could influence the conduct of science itself.
AI-enabled analytical tools mean researchers can now generate
sophisticated simulations of natural or physical systems, creating
`digital siblings' of real-world systems that can be used for
experimentation and analysis. Machine learning models that combine the
ability to learn adaptively from data with the ability to make
structured predictions reflecting the laws of nature can help
researchers untangle the web of cause-effect relationships that drive
the dynamics of complex systems. AI-assisted laboratory processes could
increase the efficiency of experiments, and support researchers to
develop and test new hypotheses.

Achieving this potential will require advances in the science of AI, the
design of AI systems that serve scientific goals, and the engineering of
such systems to operate safely and effectively in practice. These
advances in turn rely on interdisciplinary collaborations that connect
domain expertise to the development of machine learning models, and feed
the insights generated by such models back into the domain of study. As
interest in the potential of AI to drive a new wave of research grows,
the challenge for the field is to identify technical and operational
strategies to realise this potential. In the process, new questions
arise about the future of `AI for science'; whether this will emerge as
a distinct field, characterised by its own research agenda and
priorities, or whether its benefits can be best achieved through
separate, domain-focused sub-fields, which seek to integrate AI into
business-as-usual across research disciplines.

In response, this document proposes a roadmap for `AI for science'.
Synthesising insights from recent attempts to deploy AI for scientific
discovery, it proposes a research agenda that can help develop more
powerful AI tools and the areas for action that can provide an enabling
environment for their deployment. It starts by exploring core research
themes -- in simulation, causality, and encoding domain knowledge --
then draws from these ideas to propose a research agenda and action plan
to support further progress. The ideas presented are inspired by
discussions at `Machine Learning for Science: Bridging Mechanistic and
Data Driven Modelling Approaches', a Dagstuhl seminar convened in
September 2022 (see Annex 1). Abstracts from the talks given at the
seminar are shown throughout this document. These talks and the
discussions they provoked should be credited for the ideas that have
shaped it. Thank you to the speakers and participants for their
thoughtful contributions to both the seminar and the development of this
work.
