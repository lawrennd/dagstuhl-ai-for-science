\section{Connecting data to causality}\label{connecting-data-to-causality}

\subsection{Causality in science and data}\label{causality-in-science-and-data}

Most scientific endeavours have a causal element: researchers want to
characterise how a system works, why it works that way, and what happens
when it is perturbed. How researchers identify cause-and-effect
relationships varies across domains. For some disciplines, the process
of hypothesis design -- data collection -- model development provides
the core structure for interrogating how a system works. In others,
where experimentation is more difficult, researchers may rely on natural
experiments and observations to compare the response of a system under
different conditions. Those studying the Earth system, for example, have
little scope to replicate planetary conditions, so instead rely on
observational data and modelling to identify the impact of different
interventions. These different approaches, however, share a modelling
approach in which researchers provide variables to create structural,
causal models.

In contrast, machine learning proceeds by learning representations or
rules from data, based on statistical information, rather than
structured rules about how a system works (such as physical laws).
Causal inference -- the ability to identify cause-and-effect
relationships in data -- has been a core aim of AI research, in service
of both wider ambitions to replicate intelligence in machines and
efforts to create AI systems that are robust in deployment. However, in
many respects efforts to integrate causal inference into AI systems have
yet to deliver \citep{Scholkopf-representation21}.

An apocryphal story in AI tells of efforts by US researchers during the
1980s to train a computer system that could distinguish between images
of tanks from the US and USSR. The resulting system delivered high
accuracy on its training data, but failed repeatedly in practice. The
system was subsequently found to be classifying images based on their
resolution and background features -- is the image grainy? Does it
contain snow? -- rather than the tanks themselves. It found patterns in
the data that were co-incident, rather than causal. That same error has
real-world implications for the AI systems deployed today. In medical
sciences, AI systems trained to detect collapsed lungs from medical
images have been proven inaccurate, after the model was found to have
learned to detect the tube inserted into the lung to enable a patient to
breath as a response to its collapse, rather than the physical features
of the lung itself \citep{Rueckel-impact20}.
In medical sciences, deployment of such systems could put patient care
at risk. In social sciences, these AI design and data bias failures can
combine to marginalise vulnerable populations \citep{Emspak-prejudice16}.

Conversely, an understanding of the structures within data can improve
the accuracy of machine learning analyses. In exoplanet discovery, for
example, machine learning is used as a tool to detect variations in
light signals from large-scale astronomical datasets. The movement of
exoplanets around stars results in periodic changes to the light signals
from those stars, as the planet obscures them in its transit. Machine
learning can detect those signals and predict where exoplanets might be
located, but the data is often noisy. Noticing that the structure of
this noise was consistent across a number of stars, which were too
distant from each other to be interacting, researchers concluded that
instrumentation effects were distorting the data, and developed a method
to model those effects and remove them from exoplanet predictions. The
result was an efficient method for exoplanet identification that
subsequently contributed to the discovery of the first potentially
habitable planet \citep{Scholkopf-causality22}.

\subsection{Causal models as a route to advancing the science of AI and
AI for
science}\label{causal-models-as-a-route-to-advancing-the-science-of-ai-and-ai-for-science}

Many of these errors in misdiagnosing cause-effect relationships arise
from a core assumption in many machine learning methods: that data
follows an independent and identical distribution (IID). In practice,
almost all data from real-world, or complex, systems will violate this
assumption, given the interconnectedness of different variables. The
task of causality in machine learning is to create models that can
manage this violation, distinguishing between patterns in data that
simply co-occur and patterns that are causal. The resulting AI systems
would be able to solve a task in many different environments, based on
an understanding of the fundamental causal mechanisms in a
system \citep{Peters-elements17}. They would be more robust in
deployment, being less likely to make incorrect predictions as the
environment in which they operate changes, and could be more efficient
to train and deploy. They would also represent a step towards
replicating human- or animal-like intelligence, being able to solve a
task in many different environments.

In these regards, causal machine learning offers a route to balancing
the widespread utility of statistical modelling with the strengths of
physical models. Causality allows models to operate at a level of
abstraction beyond strongly mechanistic approaches, such as those based
on differential equations, moving along a continuum from mechanistic to
data-driven modelling. They provide researchers with the ability to make
accurate predictions under conditions of dataset shift (enable out of
distribution generalisation); can provide insights into the physical
processes that drive the behaviour of a system; unlock progress towards
AI systems that `think' in the sense of acting in an imagined space;
while also leveraging insights that can be learned from data, but not
otherwise detected.\footnote{For reference, see the table on page 11 of
  reference \citep{Scholkopf-causality22}.} They also offer opportunities to explore
counterfactuals in complex systems, asking what the impact of different
interventions could have been, opening a door to the development of
simulation-based decision-making tools.\footnote{Such tools may have
  particular relevance in policy. For example: \citep{Mastakouri-causal20}.}

Achieving this potential requires technical developments in a number of
directions, but can also yield more effective AI systems. Such systems
would:

\begin{itemize}
\item
  Be able to operate on out of distribution data, performing the task
  for which they are trained in environments with varying conditions.
\item
  Be able to learn how to perform a task based on relatively few
  examples of that task in different conditions, or be able to rapidly
  adapt what they have learned for application in new environments
  through transfer, one-shot, or lifelong learning approaches.
\item
  Support users to analyse the impact of different interventions on a
  system, providing explanations or ways of attributing credit to
  different actions.
\item
  Respond to different ways of transmitting information between
  individuals and groups, enabling effective communication with their
  users or other forms of cultural learning.
\end{itemize}

\subsection{From methods to
application}\label{from-methods-to-application}

Achieving the level of technical sophistication required for causal
modelling requires careful model design, based on close collaboration
between machine learning and domain scientist. The process of specifying
what to represent in a causal machine learning system involves a series
of `micro-decisions' about how to construct the model, negotiated by
integrating machine learning and domain expertise. In this regard,
causal machine learning can be a positive catalyst for deeper
interdisciplinary collaboration; model construction can be a convening
point for sharing understandings between domains. However, the level of
detail required can also be in tension with efforts to promote
widespread adoption of AI methods across research. The availability of
easy-to-use, off-the-shelf AI tools has been an enabler for adoption in
many domains. The hand-crafted approach inherent to current causal
methods renders them less accessible to non-expert users. Part of the
challenge for the field is to make such methods more broadly accessible
through open-source toolkits or effective software engineering
practices.

This tension between specification and learning also highlights the
importance of nurturing a diversity of methods across the spectrum from
data-driven to mechanistic modelling. The domain (or, how much prior
knowledge is available and what knowledge should be included), research
question of interest, and other practical factors (including, for
example, compute budget), will shape where along this spectrum
researchers wish to target their modelling efforts.

While pursuing practical applications, advances in causal inference
could help answer broader questions about the nature of intelligence and
the role of causal representations in human understanding of how the
worlds work. Much of human understanding of the world arises from
observing cause and effect; seeing what reaction follows an intervention
-- that an object falls when dropped, for example -- in a way that
generalises across circumstances and does not require detailed
understanding of mathematical or physical laws. Integrating this ability
into machine learning would help create systems that could be deployed
on a variety of tasks. The process of building causal machine learning
forces researchers to interrogate the nature of causal representations
-- What are they? How are they constructed from the interaction between
intelligent agents and the world? By what mechanism can such agents
connect low-level observations to high-level causal variables? -- which
may in turn support wider advances in the science of AI.

\subsection{Directions}\label{directions-1}

Causality in machine learning is a long-standing and complex challenge.
In the context of scientific discovery, learning strategy, model design,
and encoding domain knowledge all play a role in helping identify
cause-effect relationships.

Different learning strategies can improve the `generalisability' of
machine learning, increasing its performance on previously unseen tasks,
based on learning underlying structure of a task or environment in ways
that can contribute to broader understandings of causality. Such
learning strategies include:

\begin{itemize}
\item
  Transfer learning, taking learning from one task or domain and
  applying it in another.
\item
  Multi-task learning, enabling a system to solve multiple tasks in
  multiple environments.
\item
  Adversarial learning, to reduce the vulnerability of models to
  performance degradation on out-of-distribution data.
\item
  Causal representation learning, defining variables that are related by
  causal models \citep{Scholkopf-causality22}.
\item
  Reinforcement learning strategies that reward agents for identifying
  policies based on invariances over different conditions.
\end{itemize}

Across these new learning approaches, attempts to establish causal
mechanisms are also prompting progress in machine learning theory,
through statistical formulations of core principles \citep{Guo-causal22}.

Combining different methods can also enhance the functionality of an AI
system. For example:

\begin{itemize}
\item
  Neural ODEs have been shown to identify causal structures in time
  series data \citep{Aliee-beyond21}.
\item
  Describing causal effects as objective functions in constrained
  optimisation problems can deliver a form of stochastic causal
  programming \citep{Padh-stochastic22}.
\item
  Technical interventions \citep{Jakobsen-distributional22}
  can constrain or optimise a model towards causal outcomes. As with
  simulation design, diagnostic checks can also help identify
  cause-effect relationships by examining model outputs against `reality
  criteria',\footnote{Including syntactic, semantic, and pragmatic
    elements: \citep{Stadler-wirklichkeitskriterien90}.} which compare outputs to
  real-world results.
\end{itemize}

There are also a variety of approaches to representing existing
scientific knowledge in machine learning models, notably by specifying
the assumptions made about the world through symmetries, invariances,
and physical laws (see Figure 1).

\subsection{Talks given during this workshop session}

\abstracttitle{Causality, causal digital twins, and their applications}
\abstractauthor[Bernhard Sch\"olkopf]{Bernhard Sch\"olkopf (MPI für Intelligente Systeme - Tübingen, DE)}
\license

\begin{enumerate}
\item Desiderata for causal machine learning: work with (and benefit from) non-IID data, multi-task/multi-environment, sample-efficient, OOD, generalisation from observation of marginals, interventional.
 \item Modelling taxonomy: differential equations, causal models, statistical models.
 \item How to get from one level to the next.
\item How to transfer between statistical models that share the same underlying causal model. 
\item The assumption of independent causal mechanisms (ICM) (for example, invariance/autonomy) and sparse mechanism design. 
 \item How to derive the arrow of time from ICM and algorithmic information theory.
 \item Statistical formulation of ICM: causal de Finetti.
 \item Application to exoplanet discovery and Covid-19 vaccine scenarios.
 \item Causal representations as (a) causal digital twins and (b) AI models.
\end{enumerate}

\abstracttitle{Invariance: From Causality to Distribution Generalization}
\abstractauthor[Jonas Peters]{Jonas Peters (University of Copenhagen, DK)}
\license

Assume that we observe data from a response $Y$ and a set of covariates $X$ under different experimental conditions (or environments). Rather than focusing on the model that is most predictive, it has been suggested to take into account the invariance of a model. This can help us to infer causal structure (Which covariates are causes of $Y$?) and find models that generalize better (How well does the model perform on an unseen environment?). We show a few applications of these general principles and discuss first steps towards understanding the corresponding theoretical guarantees and limits.

\abstracttitle{Can we discover dynamical laws from observation?}
\abstractauthor[Niki Kilbertus]{Niki Kilbertus (TU M\"unchen, DE \& Helmholtz AI M\"unchen, DE)}
\license

I will start with a brief introduction to identifiability of ODE systems from a unique continuous or discrete observed solution trajectory. Then, I will provide an overview of modern approaches to inferring dynamical laws (in the form of ODEs) from observational data with a particular focus on interpretability and symbolic methods. Finally, I will describe our recent attempts and results at inferring scalar ODEs in symbolic form from a single irregularly sampled, noisy solution trajectory.

\abstracttitle{Invariances and equivariances in machine learning}
\abstractauthor[Soledad Villar]{Soledad Villar (Johns Hopkins University - Baltimore, US)}
\license

In this talk, we give an overview of the progress in the last few years by several research groups in designing machine learning methods that repeat physical laws. Some of these frameworks make use of irreducible representations, some make use of high-order tensor objects, and some apply symmetry enforcing constraints. Our work shows that it is simple to parameterise universally approximating functions that are equivariant under actions of the Euclidean, Lorentz, and Poincare group at any dimensionality. The key observation is that $O(d)$-equivariant (and related group-equivariant) functions can be universally expressed in terms of a lightweight collection of dimensionless scalars (scalar products and scalar contractions of the scarla, vector, and tensor inputs). We complement our theory with numerical examples that show that the scalar-based method is simple and efficient, and mention ongoing work on cosmology simulations.

\abstracttitle{Divide-and-Conquer Equation Learning with R2 and Bayesian Model Evidence}
\abstractauthor[Bubacarr Bah]{Bubacarr Bah (AIMS South Africa - Cape Town, ZA)}
\license

Deep learning is a powerful method for tasks like predictions and classification, but lacks interpretability and analytic access. Instead of fitting up to millions of parameters, an intriguing alternative for a wide range of problems would be to learn the governing equations from data. Resulting models would be concise, parameters can be interpreted, the model can adjust to shifts in data, and analytic analysis allows for extra insights. Common challenges are model complexity identification, stable feature selection, expressivity, computational feasibility, and scarce data. In our work, the mentioned challenges are addressed by combining existing methods in a novel way. We choose multiple regression as a framework and argue how a surprisingly large space of model equations can be captured. For feature selection, we exploit the computationally cheap coefficient of determination (R2) to loop through millions of models, and by using a divide-and-conquer strategy, we are able to rule out remaining models in the equation class. Final model selection is achieved by exact values of the Bayesian model evidence with empirical priors, which is known to identify suitable model complexity without relying on mass data. Random polynomials, and a couple of chaotic systems are used as examples.
