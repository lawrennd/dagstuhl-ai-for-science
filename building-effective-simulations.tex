\section{Building effective
simulations}\label{building-effective-simulations}

\subsection{Moving upstream}\label{moving-upstream}

Science proceeds through hypothesis, observation and analysis. For
hundreds of years, researchers have advanced the frontiers of knowledge
by collecting data, compressing those observations into a model, then
computing that model to create representations of how the world works,
generating new insights about natural and physical phenomena and
theories about the systems from which those phenomena emerge in the
process \citep{Blei-build2014}.
These mathematical models rely on numerical methods: algorithms that
help solve mathematical problems where no analytical solution is
available. Today, data collection and the basic computational tasks
involved in its analysis -- linear algebra, optimisation, simulation,
and so on -- remain consistent features of the scientific process.
Progress in machine learning, however, has changed the modelling
landscape.

`AI for science' offers a data-centric approach to modelling and
simulating the world. Operating alongside the traditional mathematical
models that are central to many disciplines, machine learning provides
data-centric analytical methods that can be integrated across the
scientific pipeline, for example enabling sophisticated simulations of
real-world systems. These simulations can be used to inform model
development, test hypotheses and shape areas of research focus, or
unlock insights from complex data.

\subsection{Nurturing a diversity of
approaches}\label{nurturing-a-diversity-of-approaches}

Simulations are a well-established tool for scientific discovery. Their
fundamental task is to allow data sampling from a model where the
differences between simulation and the real world are reduced as far as
feasible, to enable experimentation or testing of the impact of
different perturbations, while allowing some measure of simplification
of the system. Effective simulators allow researchers to move from
theory to an understanding of what data should look like.

Domains such as particle physics, protein folding, climate science, and
others, have developed complex simulations that use known theories and
parameters of interest to make predictions about the system of study. AI
for science can be brought in to speed up some of these through
surrogate models. Machine learning can complement `traditional'
approaches to scientific simulation, adding components that model the
most uncertain elements of a system to strongly mechanistic models that
might otherwise be too restrictive in their assumptions.

Much early excitement surrounding AI for science was rooted in the
reverse process, asking: instead of starting with theory, could
researchers instead start with the large amounts of data available in
many areas of research and, from that data, build an understanding of
what an underpinning theory might be? Given a set of observations, is it
possible to find parameters for a model that result in simulations that
reflect the measured data? Such simulation-based inference (SBI) offers
the opportunity to generate novel insights across scientific
disciplines.

To enable such analysis, machine learning methods are needed that can
extract insights from high-dimensional, multi-modal data, in ways that
are labour- and compute-efficient \cite{Cranmer-frontier2020}. The field of
probabilistic numerics offers a way to flexibly combine information from
mechanistic models with insights from data, solving numerical problems
through statistical approaches \cite{Hennig-probabilistic2022}. Operationalising these methods
to create effective data-driven simulations requires balancing different
model characteristics. The model's parameters must be specified to a
sufficient level of granularity to describe the real-world system, while
operating at a level of abstraction that is amenable to analysis and
computation; almost all models are `wrong' or falsifiable because of
this, but some level of abstraction is necessary to make them useful for
analysis. The simulation must also be designed to be robust, and able to
generate inferences that align with real-world observations.

\subsection{Truth, truthiness, and interfacing with the real
world}\label{truth-truthiness-and-interfacing-with-the-real-world}

The excitement underpinning AI for science stems from the aspiration to
unearth new understandings of the world, leveraging data to advance the
frontiers of knowledge. While subject to their own limitations, the
scientific community has developed checks and balances to scrutinise new
knowledge and maintain the rigour of scientific inquiry. Recent years
have seen a variety of challenges or benchmarks emerge in the machine
learning community that have come to represent the field's expected
standards of performance from algorithms on defined tasks. However,
these standards do not necessarily align with the expectations of domain
researchers.\footnote{Hermans, J., Delaunoy, A., Rozet, F., Wehenkel, A.
  and Louppe, G. (2021), Averting A Crisis In Simulation-Based
  Inference, arXiv:2110.06581 {[}stat.ML{]},
  \href{https://doi.org/10.48550/arXiv.2110.06581}{\uline{https://doi.org/10.48550/arXiv.2110.06581}}}
As data-centric simulations are integrated into scientific process,
machine learning researchers must consider their responsibility in
maintaining the integrity of the domains into which they are deployed,
raising the question: what guardrails are needed to ensure researchers
can be confident in the outputs from machine learning-enabled
simulations?

A variety of diagnostic tests can help. Core to many of these
diagnostics is analysis of whether a model is computationally faithful.
In short: the inferences generated by a simulation should reflect those
from observations.\footnote{Ibid.} One approach to checking this
alignment is to consider the consistency of distributions from inferred
and observed datasets. If the model is a good fit, the data it generates
should broadly match the data observed through experimentation.

Underpinning these diagnostics is a fundamental question about how to
manage uncertainty, in a context where different failure modes have
different implications. Put simply: when a model fails, is it worse to
be over-confident in its results, or over-conservative? In the
scientific context, over-confidence seems more likely to result in
negative outcomes, whether through giving misleading interpretations or
results or driving lines of enquiry in unproductive directions. Machine
learning methods can be designed for conservatism, reducing the risk of
false positives.

Implementing a schedule of model building, computing, critiquing, and
repeating can refine this process. One lesson from experiences of
building machine learning-enabled simulations is that there can be a
disconnect between how machine learning approaches inference and model
building, and how the same task is approached by domain scientists. From
a domain perspective, model building seems naturally an iterative
process: collect data, fit a model, find errors or areas for
improvement, update the model, and so on. This iterative process is
guided by expert intuition and knowledge; deep understanding of the
system under study and how it responds to perturbation. Machine learning
research has developed practices for prior elicitation -- using domain
knowledge to shape the structure of probabilistic models -- but the
nuances of this domain intuition are often not easily captured a priori,
instead emerging when models fail as an informal sense of what `feels'
like it should be true. This qualitative input is vital in building
effective simulations. It requires close collaboration, which in turn
requires an investment of time and energy from domain communities,
generated through mutual trust, incentives, and long-term
relationship-building.

\subsection{Connecting simulation to
practice}\label{connecting-simulation-to-practice}

Computational tools are central to the effective deployment of machine
learning-enabled simulation. The function and form of such tools must
align with the requirements of the community deploying them. Designing
computational systems to match user needs -- and work effectively in
practice -- requires both effective software engineering and close
collaboration with domain groups that can articulate the requirements
and expectations of those working in the field. To remain effective over
the longer-term, such systems must leverage effective software
engineering practices, including embedding version control and building
interfaces that work with other models and systems. Those practices, and
the software systems that emerge from them, must be designed for the
needs of those using the system, drawing from existing best practices in
software engineering, but adapting those practices to reflect the needs
of the domain for deployment.

Constructing computational tools requires a mix of technical insight and
craft skill -- of knowledge and know how. Tools produced by the machine
learning community differ in their usefulness on different problems:
some work well for certain tasks, but not for others. Without access to
such craft skills, those outside the `AI for science' community can find
it challenging to determine which tools to use for which purposes,
reducing the generalisability of existing methods and approaches. This
challenge becomes particularly visible when practitioners are tightly
integrated into the analysis pipeline, such as in applications in
developmental biology, in the developing world, and in data-centric
engineering. Widening access to the field will require user guides that
characterise which simulations are effective for which tasks or
purposes, supported by case studies or user stories that help demystify
how machine learning can work in practice.

\hypertarget{directions}{%
\subsection{Directions}\label{directions}}

Machine learning typically requires an explicit representation of a
likelihood, but these are often difficult to compute. Further advances
in SBI are necessary to allow researchers to identify model parameters
from data.

\begin{itemize}
\item
  Techniques such as likelihood-free inference can enhance existing
  Bayesian methods for inferring posterior estimations.\footnote{Alsing,
    J., Charnock, T., Feeney, S. and Wandelt, B. (2019) Fast
    likelihood-free cosmology with neural density estimators and active
    learning, arXiv:1903.00007 {[}astro-ph.CO{]},

    \uline{\url{https://doi.org/10.48550/arXiv.1903.00007}}}
\item
  Building surrogate models,\footnote{See above, and Lavin, A., Zenil,
    H., Paige, B., Krakauer, D., Gottschlich, J., Mattson, T.,
    Anandkumer, A., Choudry, S., Rocki, K., Baydin, A.G., Prunkl, C.,
    Paige, B., Isayev, O., Peterson, E., McMahon, P.L., Macke, J.,
    Cranmer, K., Zhang, J., Wainwright, H., Hanuka, A., Veloso, M.,
    Assefra, S., Zheng, S., and Pfeffer, A. (2021) Simulation
    Intelligence: Towards a New Generation of Scientific Methods,
    arXiv:2112.03235 {[}cs.AI{]},
    \href{https://doi.org/10.48550/arXiv.2112.03235}{\hfill\break
    \uline{https://doi.org/10.48550/arXiv.2112.03235}}} using Bayesian
  approaches for simulation planning to optimise information
  gain,\footnote{See, for example, Cranmer, K., Heinrich, L., Head, T.
    and Louppe, G. Active sciencing, at:
    \href{https://github.com/cranmer/active_sciencing}}
  or deploying emulations\footnote{Boelts, J., Lueckmann, J.M., Gao, R.,
    and Macke, J.H. (2022) Flexible and efficient simulation-based
    inference for models of decision-making eLife 11:e77220
    \href{https://doi.org/10.7554/eLife.77220}}
  can also enhance the efficiency of simulations.
\item
  Probabilistic numerics offers a route to develop statistically-optimal
  algorithms that are amenable to comprehensive uncertainty
  quantification, leveraging Gaussian Process-based Ordinary
  Differential Equation (ODE) solvers to pursue simulation as an
  inference problem.\footnote{Kersting, H. (2021) Uncertainty-aware
    numerical solutions of ODE's by Bayesian Filtering, available at:
    \href{https://hanskersting.github.io/publication/phd-thesis/}}
\end{itemize}

Operationalising these approaches will also require new toolkits to
support implementation of probabilistic numerical methods.\footnote{See,
  for example, the previous Dagstuhl meeting on this topic:
  \href{https://www.probabilistic-numerics.org/meetings/2021_Dagstuhl/}
  and Schmidt, J., Kramer, N. and Hennig, P. (2021) A Probabilistic
  State Space Model for Joint Inference from Differential Equations and
  Data, arXiv:2103.10153 {[}stat.ML{]},
  \href{https://doi.org/10.48550/arXiv.2103.10153}}

Computational faithfulness -- alignment of inferred parameters with
scientific knowledge -- can be achieved through:

\begin{itemize}
\item
  Diagnostic checks in the self-consistency of the Bayesian joint
  distribution, which measure the scientific quality of the regions
  computed by Bayesian SBI methods.\footnote{Hermans, J., Delaunoy, A.,
    Rozet, F., Wehenkel, A. and Louppe, G. (2021), Averting A Crisis In
    Simulation-Based Inference, arXiv:2110.06581 {[}stat.ML{]},
    \href{https://doi.org/10.48550/arXiv.2110.06581}
    and Mishra-Sharma, S. (2021) Inferring dark matter substructure with
    astrometric lensing beyond the power spectrum, arXiv:2110.01620
    {[}astro-ph.CO{]},
    \href{https://doi.org/10.48550/arXiv.2110.01620}}
  Checking for self-consistency gives a sense whether the model is `good
  enough' (ie whether the inference engine gives a good sense of the
  posterior).
\item
  Enforcing conservative neural ratio estimation through binary
  classifier specification, producing more conservative posterior
  approximations.\footnote{Delaunoy, A., Hermans, J., Rozet, F.,
    Wehenkel, A. and Louppe, G. (2022) Towards Reliable Simulation-Based
    Inference with Balanced Neural Ratio Estimation, arXiv:2208.13624
    {[}stat.ML{]},
    \href{https://doi.org/10.48550/arXiv.2208.13624}}
\item
  Hybrid modelling, which combines machine learning components learned
  from data with the mechanistic components specified by existing domain
  knowledge.\footnote{Wehenkel, A., Behrmann, J., Hsu, H., Sapiro, G.,
    Louppe, G., and Jacobsen, J.H. (2022) Robust Hybrid Learning With
    Expert Augmentation, arXiv:2202.03881 {[}cs.LG{]},
    \href{https://doi.org/10.48550/arXiv.2202.03881}}
\item
  Further study of the impact of model misspecification could also help
  generate new robustness diagnostic checks.\footnote{\href{https://arxiv.org/abs/2209.01845}{Cannon,}
    P., Ward, D. and Schmon, S.M. (2022) Investigating the Impact of
    Model Misspecification in Neural Simulation-based Inference,
    arXiv:2209.01845 {[}stat.ML{]},
    \href{https://doi.org/10.48550/arXiv.2209.01845}}
\end{itemize}

`Digital twins' have recently received much attention as a tool to
exploit sophisticated simulations. In Earth sciences, for example,
ambitious efforts to develop a digital twin of the Earth propose to
allow more accurate forecasting, visualisation, or scenario-testing of
the impact of climate change and efforts to mitigate it.\footnote{For
  example: European Commission (2022) Destination Earth -- new digital
  twin of the Earth will help tackle climate change and protect nature,
  available at:
  \href{https://ec.europa.eu/commission/presscorner/detail/en/IP\_22\_1977}}
The challenge is to integrate different models or components of a system
-- for example, connecting atmospheric models, with land models, with
models of human behaviour -- in a way that represents the complete Earth
system. That requires consideration of the different levels of
granularity with which these different models operate: economic models
of human behaviour, for example, operate with different assumptions and
levels of enquiry in comparison to physical models of ocean circulation.
The full range of granularities becomes apparent when considering that
specific applications, such as disease monitoring on poultry farms, sit
within the wider ecosystem of the natural and built environment. A
digital twin needs to make choices about what levels of granularity it
is operating at, from the scale of the poultry farm to the planet. The
questions that emerge from such ambitions is: what level of granularity
is helpful or necessary to deliver effective results? And what
interfaces between diverse models might be possible?

\subsection{Talks given during this workshop session}

\abstracttitle{Information from data and compute in scientific inference}
\abstractauthor[Philipp Hennig]{Philipp Hennig (Universität Tübingen, DE)}
\license

Simulations are central to scientific inference. Simulators are typically treated as black boxes, with the inference loop wrapped around them. This approach is convenient for the programming scientists, but can be highly inefficient. Probabilistic numerical methods represent computational and empirical data in the same language, which allows for inference from mechanistic knowledge and empirical data in one combined step. I will argue that scientific computing needs to embrace such new computational paradigms to truly leverage ML in science, which also requires rethinking scientific codebases.

\abstracttitle{ODE filters and smoothers: probabilistic numerics for mechanistic modelling}
\abstractauthor[Hans Kersting]{Hans Kersting (INRIA - Paris, FR)}
\license

Probabilistic numerics (PN) unifies statistical and numerical approximations by formulating them in the same language of statistical (Bayesian) inference. For ODEs, a well-established probabilistic numerical method is ODE filters and smoothers which can help to deal more aptly with uncertainty in mechanistic modeling. In the first half of this talk, we will first introduce PN and then present ODE filters/smoothers as a specific instance of PN. In the second half, we will discuss how ODE filters/smoothers can improve mechanistic modeling in the natural sciences and present a recent application of inferring the parameters of real-word dynamical system.

\abstracttitle{Four short stories on simulation-based inference}
\abstractauthor[Jakob Macke]{Jakob Macke (Universität Tübingen, DE)}
\license

Many fields of science make extensive use of simulations expressing mechanistic forward models, requiring the use of simulation-based inference methods. I will share experiences and lessons learned from four applications: Describing the dynamics and energy consumptions of neural networks in the stomatogastric ganglion; inferring parameters of gravitational wave models; optimising single-molecule localisation microscopy, and building computational models of the fly visual system. I will try to convey some thoughts on the challenges and shortcomings of current approaches.

\abstracttitle{Towards reliable simulation-based inference and beyond}
\abstractauthor[Gilles Louppe]{Gilles Louppe (University of Liège, BE)}
\license

Modern approaches for simulation-based inference build upon deep learning surrogates to enable approximate Bayesian inference with computer simulators. In practice, the estimated posteriors' computational faithfulness is, however, rarely guaranteed. For example, Hermans et al., 2021 have shown that current simulation-based inference algorithms can produce posteriors that are overconfident, hence risking false inferences. In this talk, we will review the main inference algorithms and present Balanced Neural Ratio Estimation (BNRE), a variation of the NRE algorithm designed to produce posterior approximations that tend to be more conservative, hence improving their reliability.

\abstracttitle{Modeling the data collection process: My journey}
\abstractauthor[Thomas G. Dietterich]{Thomas G. Dietterich (Oregon State University - Corvallis, US)}
\license

In this talk, I will describe three examples of my attempts to integrate subject-matter knowledge with machine learning. The first example involves predicting grasshopper infestations. I will sketch the methodology in which we first modeled the life cycle of the grasshoppers to capture the factors that affect their population. Unfortunately, most variables of interest were not measured, so we used the model to guide the construction of proxy variables. Ultimately, this project did not succeed, but it is hard to determine whether this is due to modeling problems or to the chaotic nature of the biological phenomenon.
