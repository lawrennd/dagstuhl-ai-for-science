This report documents the programme and the outcomes of Dagstuhl Seminar 22382 "Machine Learning for Science: Bridging Data-Driven and Mechanistic Modelling".

Today's scientific challenges are characterised by complexity.
Interconnected natural, technological, and human systems are influenced
by forces acting across time- and spatial-scales, resulting in complex
interactions and emergent behaviours. Understanding these phenomena ---
and leveraging scientific advances to deliver innovative solutions to
improve society's health, wealth, and well-being --- requires new ways of
analysing complex systems.

The transformative potential of AI stems from its widespread
applicability across disciplines, and will only be achieved through
integration across research domains. AI for science is a rendezvous
point. It brings together expertise from AI and application domains;
combines modelling knowledge with engineering know-how; and relies on
collaboration across disciplines and between humans and machines.
Alongside technical advances, the next wave of progress in the field
will come from building a community of machine learning researchers,
domain experts, citizen scientists, and engineers working together to
design and deploy effective AI tools.

This report summarises the discussions from the seminar and provides a roadmap to suggest how different communities can collaborate to deliver a new wave of progress in AI and its application for scientific discovery.