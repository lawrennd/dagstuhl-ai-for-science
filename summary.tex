This report documents the programme and the outcomes of Dagstuhl Seminar 22382 "Machine Learning for Science: Bridging Data-Driven and Mechanistic Modelling".

Today's scientific challenges are characterised by complexity.
Interconnected natural, technological, and human systems are influenced
by forces acting across time- and spatial-scales, resulting in complex
interactions and emergent behaviours. Understanding these phenomena ---
and leveraging scientific advances to deliver innovative solutions to
improve society's health, wealth, and well-being --- requires new ways of
analysing complex systems.


Artificial intelligence (AI) offers a set of tools to help make sense of
this complexity. In an environment where more data is available from
more sources than ever before --- and at scales from the atomic to the
astronomical --- the analytical tools provided by recent advances in AI
could play an important role in unlocking a new wave of research and
innovation. The term AI today describes a collection of tools and
methods, which replicate aspects of intelligence in computer systems.
Many recent advances in the field stem from progress in machine
learning, an approach to AI in which computer systems learn how to
perform a task, based on data.

Signals of the potential for AI in science can already be seen in many
domains. AI has been deployed in climate science to investigate how
Earth's systems are responding to climate change; in agricultural
science to monitor animal health; in development studies, to support
communities to manage local resources more effectively; in astrophysics
to understand the properties of black holes, dark matter, and
exoplanets; and in developmental biology to map pathways of cellular
development from genes to organs. These successes illustrate the wider
advances that AI could enable in science. In so doing, these
applications also offer insights into the science of AI, suggesting
pathways to understand the nature of intelligence and the learning
strategies that can deliver intelligent behaviour in computer systems.

Further progress will require a new generation of AI models. AI for
science calls for modelling approaches that can: facilitate
sophisticated simulations of natural, physical, or social systems,
enabling researchers to use data to interrogate the forces that shape
such systems; untangle complicated cause-effect relationships by
combining the ability to learn from data with structured knowledge of
the world; and work adaptively with domain experts, assisting them in
the lab and connecting data-derived insights to pre-existing domain
knowledge. Creating these models will disrupt traditional divides
between disciplines and between data-driven and mechanistic modelling.

The roadmap presented here suggests how these different communities can
collaborate to deliver a new wave of progress in AI and its application
for scientific discovery. By coalescing around the shared challenges for
AI in science, the research community can accelerate technical progress,
while deploying tools that tackle real-world challenges. By creating
user-friendly toolkits, and implementing best practices in software and
data engineering, researchers can support wider adoption of effective AI
methods. By investing in people working at the interface of AI and
science -- through skills-building, convening, and support for
interdisciplinary collaborations --- research institutions can encourage
talented researchers to develop and adopt new AI for science methods. By
contributing to a community of research and practice, individual
researchers and institutions can help share insights and expand the pool
of researchers working at the interface of AI and science. Together,
these actions can drive a paradigm shift in science, enabling progress
in AI and unlocking a new wave of AI-enabled innovations.

The transformative potential of AI stems from its widespread
applicability across disciplines, and will only be achieved through
integration across research domains. AI for science is a rendezvous
point. It brings together expertise from AI and application domains;
combines modelling knowledge with engineering know-how; and relies on
collaboration across disciplines and between humans and machines.
Alongside technical advances, the next wave of progress in the field
will come from building a community of machine learning researchers,
domain experts, citizen scientists, and engineers working together to
design and deploy effective AI tools.
