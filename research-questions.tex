
\section{Research
questions arising from the `AI for science research agenda' discussion
during the Dagstuhl workshop}

\subsection*{Building AI systems for science}

\begin{itemize}
\item
  How can AI systems accurately generalise from finite observations? How
  can they detect causality or structure from finite observations?
\item
  What is the computational cost of complexity, and what methods can
  help manage this?
\item
  What forms of system calibration and uncertainty quantification are
  useful in the context of scientific discovery? Are theoretical
  guarantees necessary?
\item
  What new forms of explainability or interpretability could facilitate
  the deployment of AI in science?
\item
  How could AI support generalisation from a small number of
  observations? What methods could enable few- or one-shot learning?
\item
  How can AI researchers build meaningful models from data to accurately
  represent causal mechanisms in the system of study? How can
  researchers identify the most effective model for their system of
  study?
\item
  What does it mean to understand a model? How can researchers combine
  explainability with complexity?
\item
  How can AI methods be made robust and easy to use in deployment by
  domain scientists?
\item
  How can advances in simulation methods be applied in domains where the
  system at hand is less easily described by equations?
\item
  What advances are needed to expand the use of simulations in science?
  How can AI help simulate laboratory experiments or environments,
  helping make more efficient different elements of the scientific
  process? How might this be expanding in the long-term, for example to
  planning experimental design or helping identify where data is
  missing?
\item
  How can `digital siblings' be used to explore the impact of different
  interventions on complex systems?
\end{itemize}

\subsection*{Combining human and machine intelligence}

\begin{itemize}
\item
  How can AI researchers best extract, formalise and assimilate the
  knowledge that domain researchers have acquired? What forms of
  knowledge representation can formalise scientific understandings of
  the world, translating these to objective functions for AI systems?
  What forms of human-AI engagement can make use of the `qualitative'
  knowledge -- or intuitions about a system -- that domain researchers
  have accumulated?
\item
  How can AI capture the qualitative understanding that researchers have
  of their domain to more accurately or effectively characterise a
  system?
\item
  How can AI be effectively deployed to mine the existing research
  knowledge base -- for example, papers, databased, and so on -- to
  extract new insights?
\item
  Where can automation support research progress? Which elements of the
  scientific process could be automated, and where is human input vital?
\item
  What forms of collaboration are needed to effectively specify helpful
  outputs from an AI system?
\item
  How can insights from AI analysis be returned to researchers in an
  actionable way? What mix of AI design, engineering, social
  interaction, and education can make effective interfaces between
  domain researchers and AI systems?
\item
  How can the outputs of AI systems be made interpretable for scientific
  users?
\item
  How can AI researchers better understand and design for the forms of
  interpretability that resonate with domain researchers?
\item
  What processes of collaboration or co-design can help describe what
  scientists `need to know from an AI system?
\item
  What best practices or methods can be deployed to effectively
  communicate uncertainty from AI systems to human users?
\end{itemize}

\subsection{Influencing practice and adoption}

\begin{itemize}
\item
  What are the craft skills in AI for science? What `know how' is
  necessary to make AI work effectively in practice?
\item
  What skills-building or forms of outreach can help take AI tools out
  of the AI community and into `the lab'?
\item
  How has machine learning been used most effectively for research and
  innovation? What best practices, or lessons, do existing efforts in AI
  for science offer?
\item
  Which AI tools are suitable for which purposes, disciplines, or
  experimental designs? Is it possible to create a taxonomy for science?
\item
  Are there generalisable methods or conclusions that can be taken from
  domain-specific efforts to deploy AI for science?
\end{itemize}
